
\documentclass{acm_proc_article-sp}
%\documentclass{sig-alternate}
\usepackage{textcomp}

\begin{document}

\title{Function Similarity Evaluation Tool}
\subtitle{CSEE E6861y - Computer-Aided Design of Digital Systems}

\numberofauthors{1}
\author{
\alignauthor
Kshitij Bhardwaj \ \ \ \ Paolo Mantovani\\
{\small kb2673@columbia.edu \ \ \ \ \ \ \ pm2613@columbia.edu}
}

\maketitle

\begin{abstract}
This tool evaluates the similarity of two functions, which have the same support,
given two covers representing the functions. The tool kernel is a divide and
conquer algorithm which recursively splits the covers with respect to a variable
by producing positive and negative cofactors, until a termination rules applies.

The program is written in \texttt{C++}, which compiles to fast native executable
and the code consists of a main function, which calls the basic components of
the program:
\begin{itemize}
  \item {\bf The parser}: takes two PLA files as input, which describe the two
    covers, each of which has the same number of input variables and a single
    output;
  \item {\bf Function Similarity}: is the heart of the program, containing the
    recursive function. The latter takes the parsed covers and performs three
    basic steps:
    \begin{enumerate}
      \item Chek if a termination rule applies; It is worth to notice that the
        rules are checked in order of complexity, so that the faster rules will
        match first. Also, most of the rules have running time linear in the
        number of literals or in the number of cubes in the covers.
        We implemented alaso a few rules with quadratic complexity, however,
        these are not enabled by default.
      \item If not, then split the covers and do two recursive calls on the
        positive and nefative cofactor;
      \item If yes, then evaluate the similarity of the current cofactors and
        return.
      \item During the recursion, if the \texttt{verbose mode} is enabled, all
        the intermediate PLA files are saved to disk. The local variables and
        the arguments passed to the recursive function are small enough not to
        incur in a memory fault.
    \end{enumerate}
  \item {\bf The printer}: is responsible for invoking the methods for printing
    the recursion tree and the PLA files if \texttt{verbose} is active.
\end{itemize}

\end{abstract}


\section{Function Similarity}

The similarity between two functions represents the percentage of input patterns,
among the complete set of possible input vectors, for which the functions give
the same output.
For this tool we assume that the functions are completely speciefied (i.e. the
don't care set, or DC-set, is empty).
A brute force approach to this problem, involves the generation of the truth
tables, starting from the covers, which has exponential complexity. A classic
solution is to follow the ``Divide et Impera'' technique, that means to split
the problem into smaller subproblems until a trivial and fast termination
step can be applies.

\subsection{Shannon Decomposition}

When no termination rule matches the current covers, the program produces
cofactors to split the problem. The following equation shows how we can
recover the similarity of the original functions, given the similarity of the
cofactors:

\begin{equation}
  \label{eq:shannon}
  FSim(F, G) = \frac{FSim(F_{x}, G_{x}) + FSim(F_{x'}, G_{x'}}{2}
\end{equation}

Notice that the function $FSim$ takes covers as input, but returns the similarity
of the functions they represent. Variable $x$ is a generic splitting variable.
This simple rule is always correct, because when we split the cover, with respect
to a variable, we obtain two cofactors which no longer depend on the splitting
variable and contain informations about one half of the original cover.

\subsection{Termination Rules}

A handful set of termination rules is key to speed up the program and reduce
the number of recursive calls. However, too complex rules have been discarded,
because the cost for checking if they match is often times higher than the cost
of an extra recursive call.

\begin{itemize}
  \item {\bf B1}: Both covers are empty. This is the simplest rule and we just
    return 1.
  \item {\bf B2}: Both covers are a tautology. Again we return 1, however,
    tautology checking can be quite expensive, therefore, we chose not to
    implement the complete recursive algorithm. Instead, we just try to mathc
    tautology checking termination rules and let our main recursion continue.
    In fact, if one of the covers is truly a tautology, this rule should probably
    match anyway within a few recursion, while if none of the cofactors ends
    up being a tautology, we waste lot of time doing recursive cofactoring on
    covers, just to prove that they are not a tautology and we cannot terminate
    the main recursion.
  \item {\bf B3}: One cover is empty. In this case we just need to return the
    size of the OFF-set of the non empty cover, divided by the total number of
    possible input patterns. To get the OFF-set cardinality, we actually count
    the number of elements in the ON-set, because the given covers contain cubes
    which cover the ON-set. This operation is not trivial and in the most general
    case requires to apply again the divide and conquer technique on the non
    empty cover. This time doing a recursion to calculate the ON-set cardinality
    is worth the effort, because this rule terminates the main recursion.
  \item {\bf B4}: One cover is a tautology. This rule is complimentary to rule B3
    and we need to return the ON-set cardinality of the cover which is not a
    tautology, divided by the number of possible imput vectors.
  \item {\bf B5}: One cover is empty and the other is a tautology. This rule gives
    a trivial answer, which is that the similarity to be returned is 0.
  \item {\bf B6}: Both covers have a single cube. In this case we can easily
    calculate the cardinality of the ON-sets of the two covers, by simply
    looking at the number of don't cares in the input part of the cubes.
    Moreover, we can quickly calculate the intersection of the two cubes, thus
    deriving in linear time both common zeros and common ones. The sum of common
    zeros and common ones, divided by total number of minterms gives the similarity
    of the covers.
  \item {\bf B7}: One cover has a single cube and g has non intersecting cubes.
    Assuming we can check that a cover has only non intersecting cubes, then this
    rules extends rule B6: the size of the ON-set of the cover with multiple
    cubes is obtained by summing up the number of minterms covered by each of
    its cubes, while the intersection between the two covers is calculated by
    intersecting the single cube of the first cover, with every cube of the
    other. Notice that if the cubes in the second cover had intersections, then
    complexity would be exponential.
    Unfortunately, the problem of checking wether the cubes intersect or not
    has quadratic complexity, thus we chose to let this rule as an option, which
    can be enabled by adding the flag \texttt{--single\_disjoint} to the command
    line when starting the tool.
  \item {\bf B8}: Both covers have multiple non intersecting cubes. This rule
    extends further rules B6 and B7. Intuitively this is the most powerful
    of the three and it could save many recursive calls at the cost of two
    quadratic checks. Again we chose to leave it as an option, enabled by
    the flag \texttt{--multi\_disjoint}. When this rule is enabled, rule B7
    is enabled as well.
  \item {\bf B9}: Both covers show single input dependence with respect to
    the same variable (same value). If this rule matches it means that the
    covers are not tautology therfore the two covers look exactly the same,
    but for the number of rows. Notice that there is no guarantee that the
    initial covers or the cofactors don't have duplicated or redundant
    cubes. In this case we obviousely return 1.
  \item {\bf B9}: Both covers show single input dependence with respect to
    the same variable (but different value). This is complementary to rool
    B9 and we just return 0.
  \item {\bf B11}: Both covers show single input dependence, but with respect
    to different variables. In this case it does not matter whether the variable
    is shown in its positive or complemented form. The similarity of the
    two functions is exactly 0.5, because in this special condition the two
    covers have half of in ON-set in common and half of the OFF-set in common.
  \item {\bf B12}: One cover shows single input dependence. This rule is not
    trivial, but it allows early termination, by performing one last cofactoring
    with respect to the variable on which one cover has single input dependence.
    In fact, we obtain one empty cofactor and a tautology, thus rules B3 and
    B4 apply on the cofactors.
\end{itemize}

A wise ordering of the rules checking can speed up the algorithm in most cases,
because rules with a fast check and, even more important, which allow to terminate
quickly, will match first. Also, we combine the checking of similar rules, to
avoid repeated calculations and we chose the following ordering:
\begin{enumerate}
  \item Check rules from B1 to B5;
  \item Check rules from B9 to B12;
  \item Check rules from B6 to B8 (depends on which are enabled);
  \item Check unateness conditions for termination or pruning. This set of rules
    is presented in the next section.
\end{enumerate}

\subsection{Unateness Conditions}

For function similarity, having simply a condition of unateness, does not help to
improve the speed of the algorithm, however, there are a few special cases, which
allows to prune the recursion tree.

\begin{itemize}
  \item {\bf U13}: Both covers are positive/negative unate on the same variable with
    no DCs. In other words, this means that both covers have either all 1's or all
    0's on the same culumn. If we chose the variable bound to that column, then
    rule B1 applies to a pair of cofactors and we can prune one branch of the
    recursionon.
  \item {\bf U14}: Both covers are unate on the same variable with no DCs, but one
    is positive unate, while the other is negative unate. This is a termination rule
    because it implies that the covers do not share any ON-set minterm; thus we just
    need to calculate the cardinality of the two ON-sets to obtain the number of
    common zeros, which divided by the total number of combinations gives the
    similarity of the functions.
  \item {\bf U15}: One cover is positive/negative unate in a variable Xi with no Dcs.
    If we pick Xi as splitting variable then rule B3 applies to a pair of cofactor
    and we can therefore prune one branch of the recursion.
\end{itemize}

Again the ordering is important, therefore we check first if rules U14 applies,
because it terminated the recursion.

\subsection{Splitting Variable}

When none of the rules above applies, we pick a binate variable as splitting
variable, to try keeping the tree balanced. This choice aims to favour rule B6,
which is very quick to check and requires a simple calculation: getting the
similarity of two cubes.

\section{Regression Test}

To test our program we implemented a script in the \texttt{Makefile} that
executes the program with a bunch of input PLAs and checks the result against a
golden output, which we calculated manually for most of the inputs.
The test suite includes also a larger example, with 20 input variables,
which we couldn't test manually.

Other trivial input files, not included in the test suite, have been used to
test the rules on by one, right after we implemented it.

The regression script allowed us to build the program in an incremental way,
checking that any additional feature we added did not compromise the
previous functionality.


\subsection{Covers with a large support}

Our program can easily handle large input files, with covers that have many
cubes. However, a major constraint is given with the number of variables.
The total number of input patterns, in fact, is exponential in the number
of variables, but the integer type, for instance, is encoded with 32 bits,
thus we can'r represent numbers larger than $2^{32} - 1$.
A second issue that could occur when we have a large number of input variables
is that the similarity could be too close to 1 or 0, thus it can't be represented
even using double precision floating point numbers.

\section{Conclusions and Possible improvements}

The tool we propose performs a fast Function Similarity Evaluation starting
from two given covers in PLA format which share the same support.
As a further improvement to the tool we envision to implement a functio
that performs SCC over the covers. This step could be expensive, but
removing all redundant cubes could dramatically reduce the total
recursive calls, thus speeding up the algorithm. Given the complexity
of SCC, the user should be able to set a threashold on the size of the
covers, so that the tools performs it only when the cover is small
enough. This is a semplification ruele ({\bf M1}) which was not a
requirement for this version of the tool.

%\bibliographystyle{abbrv}
%\bibliography{ref}

\end{document}
