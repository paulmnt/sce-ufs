
\documentclass{acm_proc_article-sp}
%\documentclass{sig-alternate}
\usepackage{textcomp}

\begin{document}

\title{Function Similarity Evaluation Tool}
\subtitle{CSEE E6861y - Computer-Aided Design of Digital Systems}

\numberofauthors{1}
\author{
\alignauthor
Kshitij Bhardwaj \ \ \ \ Paolo Mantovani\\
{\small kb2673@columbia.edu \ \ \ \ \ \ \ pm2613@columbia.edu}
}

\maketitle

\begin{abstract}
This tool evaluates the similarity of two functions, which have the same support,
given two covers representing the functions. It

The main purpose of this project is to evaluate the efficiency and
performance improvements that can be obtained by integrating hardware
accelerators within a system framework, based on a tiled Network on Chip (NoC)
and designed using High Level Synthesis (HLS) tools.
In particular, we show that a hardware accelerator, synthesized from
SystemC and integrated into the SoC, can outperform a processor, with comparable
parallelism, executing the same algorithm in software.
We report experimental results for both level of abstraction: System level and
RT level.
\end{abstract}

\keywords{SoC Design, Hardware Accelerator, High Level Synthesis}

\section{Introduction}
It is well known that hardware accelerators are widely integrated in modern
embedded systems. The classic RTL design flow, though, does not seem to be
suitable any more for a fast design of a complex and heterogeneous system.
In this project we exploit the case study of an image processing algorithm
(the \emph{JPEG decoder}) to evaluate a complete design flow, which begins
form a specification, written by a third party in a high level language, and
finishes with a synthesizable RTL hardware implementation of the same algorithm.
Intermediate steps allow to validate the design at different level of abstraction
and to estimate its performance within a complete system framework.

 
The structure of the report follows the phases of the project.
In section \ref{sec:nof} we start by analyzing the system framework provided.
Section \ref{sec:jpeg} introduces the JPEG decoder and the CHStone \cite{chstone}
benchmark suite. Then we briefly describe the porting from plain C to SystemC and 
the accelerator structure. At the end of the section we report the results obtained
by running Cadence HLS tool \emph{CtoS}.
In section \ref{sec:eval} we report the experimental results and finally in section
\ref{sec:end} we conclude with the lessons learned and with strengths and weaknesses
of this design flow.

\section{SoC Framework for {\subsecit plug \& play} }
\label{sec:nof}

The starting point of the project is a tiled based NoC designed in SystemC.
It can host multi-core processor tiles, composed by 4 ``MIPS like`` cores with level
1 and level 2 caches, described at the RTL, and hardware accelerator tiles,
with dedicated ``scratch pad'' memory. Each tile has also a Network Interface (NI),
which provides a uniform protocol to connect third party accelerators, assuming
that they are wrapped within a Transaction Level Modeling (TLM) module, with TLM FIFO
for data I/O transfers. The NI handles the communication through the NoC by implementing
software primitives for explicit message passing communication among the tiles.
The processor tiles have therefore the capability to delegate the available
accelerators to complete certain tasks. The details about the network protocol
and the implementation of the tiles are described in \cite{cota}.

\section{CHStone JPEG decoder}
\label{sec:jpeg}

The algorithm chosen to be accelerated is the JPEG decoder from the CHStone
\cite{chstone} benchmark suite. The choice of the CHStone is motivated mainly by
the fact that these benchmarks are explicitly meant to test the efficiency of
HLS tools. Also, the source code is plain C and it includes the
test vectors, therefore it can be easily compiled for the MIPS like core
available within the NoC, which doesn't support an OS. Among the algorithms of
the suite, the JPEG was the best candidate, because of its complexity.
Considering, in fact, the communication overhead deriving from the NoC and the
NI, we must assume that an accelerator will be more efficient, in terms of
performance vs. power, only if the computational complexity of the task is large
enough. For small and simple algorithms, instead, a pure software
implementation, which does not require data transfer through the NoC is expected
to be more efficient.
Moreover we derive the TLM SystemC accelerator from the C code,
because existing implementations of SystemC accelerators are usually not
suitable for being executed in software, without strongly modifying the kernel
of the algorithm. Our purpose, in fact, is not to build a state of the art
accelerator, but to prove that through this new design flow we can increase
productivity and performance, guaranteeing at the same time a significant power
saving.

\subsection{Software implementation}

We shortly describe CHStone JPEG decoder structure. The first function call
parses the information contained in the JPEG header, which has undefined size.
It is therefore compulsory to load all the coefficients and to build a few tables
before starting the decoding process.
Special characters, called markers, identify the beginning of different sections,
which provide basic information about the picture (such as the size), the coefficient
used for compression and quantization, quality parameters and finally the
compressed data. The other functions implement the baseline JPEG decoder and
use the information provided in the header to reconstruct the original RGB
table (notice that the compression is ``lossy``). The complete JPEG standard is included
in \cite{jpeg}.

A small test vector was statically included within the CHStone code.
In order to obtain significant experimental results, however, we wrote a small
preprocessor that parses different and bigger images and outputs a configuration
header for the test bench used for simulations.

\subsection{SystemC implementation}

In this section we list the major changes applied to the CHStone code, during the 
porting process to synthesizable SystemC.
Fig.~\ref{fig:jpgdec} is a block diagram representation of the SystemC code, intended
to help in understanding the interaction between the processes and how the accelerator is
interfaced to the standard TLM FIFO for being plugged to the NoC.
Porting to SystemC included the following steps:
\begin{itemize}
 \item Split of the sequential C function into smaller concurrent behaviors, which
are synchronized through \emph{sc\_signals} and a handshake protocol.
Data transfers are enabled by shared local memories which act as temporary scratch pads.
 \item Conversion of global variables into shared attributes of the module,
which are accessible by every process declared within the System-C module.
 \item Replacement of all the pointers with array indexes declared as member variables.
This allows the HLS tool to parse and build the code correctly.
 \item Enabling the overlap of I/O transfers and computation. This allows to decode bigger
images, because the picture does not need to be stored entirely within th accelerator.
 \item Generation of the output file header.
 \item Implementation of the TLM wrapper to connect the accelerator to its test
bench. The TLM interface is also required to integrate the accelerator into the
NoC.
 \item Implementation of a configurable test bench, which can feed the
accelerator with different input images and checks the output, with respect to a
golden output provided.
 \item Optimization of loops to enable further options for design space
exploration with the HLS tool.
\end{itemize}

\subsection{From SystemC to RTL}

\begin{table}[b]
\centering
\begin{tabular}{|l|rrrr|}
\hline
 & \textbf{Memory} & \textbf{Logic} & \textbf{TLM} & \textbf{Total} \\
\textbf{area $mm^{2}$} & 0.9786 & 0.1139 & 0.0099 & 1.1020 \\ 
\textbf{power $mW$}    & 47.495 & 274.99 & 1.3222 & 323.81 \\
\hline
\end{tabular}
\caption{\footnotesize High Level Synthesis of the JPEG decoder, constrained with a 1GHz clock frequency}
\label{tab:rc}
\end{table}

\begin{table*}[t]
\centering
\begin{tabular}{l|rcc|c|ccc}
\hline
\textbf{Image} & \textbf{Weight} & \textbf{Size} & \textbf{Compression} & \textbf{RTL ACC only} & \textbf{VP only} & \textbf{VP with ACC} & \textbf{VP Speedup} \\
\hline
chstone & 5.1K   & 90x59     & 32.67\% & 327521   & 12799005   & 9707851   & 1.31\\ 
lena    & 6.6K   & 256x256   & 3.44\%  & 2588629  & 67969104   & 24297449  & 2.79\\
fire    & 12.0K  & 340x340   & 3.54\%  & 4671280  & 122417017  & 38520484  & 3.17\\
orange  & 27.0k  & 800x496   & 2.32\%  & 15476731 & 393519047  & 110264509 & 3.56\\
moon    & 40.0k  & 900x900   & 1.69\%  & 31481890 & 784515821  & 218384556 & 3.59\\
fish    & 107.0k & 1152x864  & 3.67\%  & 39859785 & 1037381360 & 270253629 & 3.84\\
flower  & 85.0k  & 1600x1200 & 1.51\%  & 73838862 & 1830097863 & 502503637 & 3.64\\
hdcol   & 166.0k & 1920x1080 & 2.73\%  & 81696290 & 2086998030 & 550907798 & 3.78\\
\hline
\end{tabular}
\caption{\footnotesize Jpeg decoder tests, using a set of images, ordered by size.
We report latency, in terms of clock cycles for the RTL version, simulated through
CtoS, for the CHStone software, running on the ARM modeled in the Rabbits VP, and
for the accelerated version, also running on the Rabbits VP.}
\label{tab:lat}
\end{table*}

After coding and testing the accelerator with the fast SystemC simulations,
we used our code as input for CtoS, which is able to generate RTL verilog.
CtoS implementation was set to be ''target aware'', thus the tool is allowed
to add resources taken from the standard cell library and states (i.e. it increases
the latency), in order to meet the timing constraints.
Since we aim to integrate the accelerator into the existing SoC, the given constraints
are the technology (\emph{32 nm PD-SOI}) and the target clock frequency (1 GHz).

The most difficult aspect in configuring the HLS is dealing with arrays which need to
be mapped as static ram blocks, when their size is not feasible for registers.
All the buffers, drawn in light blue in Fig.~\ref{fig:jpgdec} have been mapped
to vendor static ram memories, whose liberty file and verilog view are provided by
a memory generator for the same technology used for synthesis.

The need of a significant number of memories, reduces the possibility for design space
exploration, because the HLS tool is very conservative while scheduling memory accesses.
Also memory ports are limited and the buffer chain represented in Fig.~\ref{fig:jpgdec} was
carefully designed to avoid conflicts between processes for memory accesses.

After scheduling we ran a more precise synthesis tool on the generated RTL and we obtained
area and performance reported in Table~\ref{tab:rc}. The most relevant information provided
is that almost 89\% of the area is memory. This is actually common for hardware accelerators,
because they generally target a single application which operates on large amount of data.

\section{Evaluation}
\label{sec:eval}

In this section we first focus on the high-level simulation of the accelerator,
integrated in a modified version of the Rabbits Virtual Platform (VP) framework
\cite{rabbits, cota}. The platform models the SoC described in Sec.~\ref{sec:nof},
even if the available processor model for Rabbits is a state of the art ARMv6 core,
instead of the ``MIPS R2000'' provided at the RTL level.
Actually for our case study the two processors reported a similar IPC, probably
because of the small instruction level parallelism offered by the algorithm.
Also, we are aware of the fact that RTL simulation is cycle accurate, whereas VP is 
not.
Nevertheless simulating on Rabbits allowed us to compare the process of decoding
big images, taking into account the overhead of a real system (i.e. OS, device driver,
communication, memory latency). Such a simulation, when the picture size grows,
is not feasible at the RTL level, because the running time explodes.

We then report data concerning the RTL simulations.

Table~\ref{tab:lat} summarizes simulation results with respect to the JPEG weight, the image
size and the compression ratio.

\subsection{System Level}

Firstly we try to obtain a baseline as a comparison for our JPEG decoder evaluation.
Therefore we compile the CHStone original C code for the VP ARM processor and we
run it for different image sizes and compression ratios. The column \emph{VP only}
in Table~\ref{tab:lat} reports the cycles measured\footnote{Rabbits is not cycle accurate}
during this simulations.

The setup for the next step is the following.
In addition to the TLM interface, we include in our SystemC a few configuration registers,
accessible by the NI, which will be written by a message sent from the processor
tile, when the core is asking the accelerator for some service.
Similarly, the core can poll another register to check whether new data are available
to be retrieved and stored in its local L2 cache.
Another small logic block, required to plug in the accelerator is a data width adapter.
Its task is on one hand to collect data from the messages' payload and to build the 64
B data token for the JPEG decoder when feeding its input. On the other hand, this logic
receives the output data token from the decoder and must prepare the message payload for
the NI.

We then need to write and compile a piece of code for the ARM processor.
This software routine initializes the accelerator and starts transferring
data from memory to the decoder, using DMA. A second thread polls the
accelerator, waiting for output data to be ready.
To enhance simulation speed, the output is retrieved and then discarded,
without saving the bmp file.

Fig.~\ref{fig:lat_vpacc} presents the latency of the accelerated JPEG on the VP.
Clearly the number of clock cycles grows linearly with the image size, which is
consistent with the expected asymptotic running time of the algorithm.
The offset is attributable to the initialization and communication overhead.
The graph shown in Fig.~\ref{fig:speedup_vpacc} displays finally the speedup
of the accelerated simulation with respect to the baseline described above.
Although the trend seems to be logarithmic, the points deviate significantly
from the curve. The reason is because compression ratio (JPEG weight) depends
also on the specific information and features of the original picture.
Obviously, the more details have to be encoded, the less efficient will be 
the algorithm.
A second issue that deserves attention is the fact that when the size of the
picture grows too much, while the compression ratio is very low (good compression),
the overhead for moving data can dominate the overall latency.

As an anticipation to the next section, we remark that the reported speedup
is a lower bound, because we do not capture all the micro-architectural optimizations
included in the final RTL implementation.



\subsection{Register Transfer level}


The setup for the RTL simulation should be similar to the one described in the previous
section for the System level. The system and the hardware wrapper for the accelerator
are still conceptually the same, but the simulation at this level is cycle accurate
and running time is considerably slow. Therefore we were able to run the software
CHStone on the processor tile only using the smallest image available. The co-simulation
of the accelerator and the rest of the system was split into two separate simulations:
using a dummy accelerator, that fetches the input and loops back to the output, we measured the
communication overhead for initializing the accelerator, sending and receiving the packets;
from CtoS console, we then ran the RTL simulations of the decoder with different pictures, but
connecting a test bench to the TLM FIFO. Also, being the processor different, we needed to
recompile the CHStone for the baseline and the software routine used with the dummy accelerator.

The graph in Fig.\ref{fig:lat_rtl} shows the growth of the latency, with respect to the
picture size. Not surprisingly, we obtain again a linear trend, but it is worth to notice
that the coefficient measured for the System Level simulation is more than six times larger.
This is due to two main reasons: first it confirms that the micro-architectural optimization
allowed by the HLS tool is relevant to the performance gain achievable; second we must
consider that the test bench is able to push inputs to the queue at the maximum sustainable
rate for the accelerator, while the VP might not. We couldn't measure the communication
throughput between the core and the accelerator, but we can assume that occasionally the
decoder could be stalled, waiting for the next data token.


As mentioned, we could measure the running time for the software version only with the
smallest picture, which is the one included in CHStone code. Thus we could calculate
the accelerator speedup for this picture, which is 18.4\texttimes. Clearly the RTL implementation
exploits micro-architectural optimization, but if we were able to simulate different
inputs, we could better analyze the impact of the picture size and check
whether the speedup still grows with a logarithmic trend. Enhancing simulation speed
for the overall SoC is left as a future work.
At the RT level we were also allowed to measure the energy consumed by
the accelerated execution, including both processor and accelerator tiles energy,
with respect to the energy of software simulation with the processor only.
The energy saving is remarkable and stands above 90\%. It is also worth to mention
that about two thirds of the overall energy is imputable to communication.


\section{Conclusions}
\label{sec:end}

In this project we implemented a hardware accelerator for JPEG decoding in SystemC.
We exploit this case study in order to evaluate a design flow that takes advantage
of a preexisting SoC infrastructure, also designed in SystemC and equipped with an RTL
implementation of a processor tile. The design was then synthesized through a HLS
tool, in order to estimate area and power consumption. Finally we measured through simulation
the speedup at different level of abstraction (and therefore different accuracy),
proving the effectiveness of the design flow and the potential benefits achievable
with a heterogeneous SoC with hardware specific accelerators.

Despite the well promising results, a weakness of this flow consists in the slow RTL
simulation, which do not allow a complete analysis of performance for large test vectors.
Also the HLS tool is not enough flexible when dealing with memories and design optimization
was considerably limited. Finally we should mention that many features of SystemC,
which enhance simulation speed and allow a quick porting of the code from another
high level language, are lost when the designer needs to target the synthesizable subset of
SystemC.


%\bibliographystyle{abbrv}
%\bibliography{ref}

\end{document}
