
\documentclass{acm_proc_article-sp}
%\documentclass{sig-alternate}
\usepackage{textcomp}

\begin{document}

\title{Retiming Tool}
\subtitle{CSEE E6861y - Computer-Aided Design of Digital Systems}

\numberofauthors{2}
\author{
\alignauthor
Kshitij Bhardwaj \ \ \ \ Paolo Mantovani\\
{\small kb2673@columbia.edu \ \ \ \ \ \ \ pm2613@columbia.edu}
}

\maketitle

\begin{abstract}
In the context of logic synthesis, the retiming step is performed on a multi
level logic network, in order to rebalance combinational paths and to squeeze
some slack to increase clock frequency.
The tool takes as input a Control Data Flow Graph (CDFG) encoded as a
Synchronous Logic Network Graph, which will be addressed simply as graph or
network through the rest of this report.

The tool has two working modes: \texttt{min cycle} and \texttt{min area}.
The former mode aims to find the minimum feasible clock cycle and retimes
the network regardless of the registers count. The latter, instead, takes
as input a target feasible clock cycle and tries to retime the network so
that the clock cycle is met and the number of registers is minimized.
\end{abstract}


\section{Tool overview}

The program is written in \texttt{C++}, which compiles to fast native executable
and the code consists of a main function, which calls the basic components of
the program:
\begin{itemize}
  \item {\bf The parser}: takes  one text input file which describes the graph,
    by listing nodes delay (combinational logic) and edges weight (registers
    current position). The parser allows the presence of comments at the end of
    each line.
  \item {\bf Min Cycle}: is the default running mode, which relies on the
    following basic algorithms \cite{retime}, which are invoked by the main function:
    \begin{enumerate}
      \item {\bf WD}: computes all paths delays and weights for each vertex pair
        and stores them into two matrices (W and D). This step is an adapted
        version of \texttt{Floyd Warshall} algorithm \cite{algo}.
      \item {\bf Binary search}: picks values from D matrix as possible target
        clock cycles.
      \item {\bf FEAS}: for each feasible target cycle, this step returns the
        retiming vector and the retimed network. The algorithm repeats the
        following steps n times, where n is the number of nodes in the network.
        \begin {itemize}
          \item {\bf Clock Period (CP)}: computes the arrival time of each vertex, given
            the current register placement.
          \item {\bf Retime}: determines the next retiming vector, based on late
            nodes, retime then retimes the network.
        \end{itemize}
    \end{enumerate}
  \item {\bf Printer 1}: is responsible for invoking the methods for printing
    the initial network information, the retiming information and the minimium
    cycle achieved. When \texttt{verbose} flag is enabled, all intermediate steps
    of WD and FEAS are printed to file.
  \item {\bf Min Area}: takes the same input file described above, plus the
    target clock cycle. This mode performs the following basic steps:
    \begin{enumerate}
      \item {\bf WD}: same as above.
      \item {\bf SIMPLEX}: given the information found in the previous step, the
        retiming problem is expressed as a set of linear disequations and a
        function to minimize, which is the total number of registers after
        retiming. The disequations are the constraints to which the network is
        subject to: there must be no edge with negative weight and the final circuit
        has to meet the target clock cycle.
        SIMPLEX class includes, among the others, the following important methods
        \cite{simplex}:
        \begin {itemize}
          \item {\bf Make Tableau}: transforms the constraints and the objective
            function into a matrix.
          \item {\bf Phase 1}: performs rows operations to get a feasible solution
            to the linear system.
          \item {\bf Phase 2}: searches for the legal and feasible retiming vector
            that minimizes the registers count. This is done again by performing
            row operations on the matrix or \texttt{tableau}.
        \end{itemize}
    \end{enumerate}
  \item {\bf Printer 2}: is responsible to print the initial network information,
    the retiming information and the minimum area achieved. When \texttt{verbose}
    flag is enabled, all intermediate steps of SIMPLEX are saved to file.
\end{itemize}

Further details on the algorithms are provided in sections \ref{sec:struct} and
\ref{sec:tech}.


\section{Data structures and algorithms}
\label{sec:struct}

This section describes in details the data structures used for each part of the
tool and gives some insights on the way the main algorithms have been coded.

\subsection{Storing the synchronous network}

To store the graph we designed a class which can serve both modes of operation
of the tool and allows for a clean and readable programming style. Also through
a massive usage of pointers, we avoid having multiple copies of the network,
thus reducing the memory requirements.
This class is named \texttt{sng} and a part from a string, which stores the name
of the circuit, it contains two main member variables: a vector of pointers to
edge and a vector of pointers to vertex. Edge and Vertex are the name of two
classes, which keep all the information necessary to run the algorithms of the
tool and to print intermediate and final results.

{\bf Vertex} includes the following items:
\begin{itemize}
  \item {\bf in}: is a vector of pointers to edge that contains a pointer per
    each edge connected and directed to the vertex.
  \item {\bf out}: is a vector of pointers to edge that contains a pointer per
    each edge departing from the vertex.
  \item {\bf id}: is the name of the node assigned based on the input file.
  \item {\bf delay}: is the combinnational delay of the node.
  \item {\bf delta}: is the arrival time, including the delay of the node,
    which is set by CP to determine the current clock cycle and which nodes
    are late.
  \item {\bf color}: is a flag used when walking the tree to perform Depth
    First Search (DFS) on the graph.
  \item {\bf c}: is the difference between the vertex in-degree and out-degree,
    which is used to generate the objective function to minimize for the
    \texttt{min area} mode.
\end{itemize}

{\bf Edge} includes the following items:
\begin{itemize}
  \item {\bf id}: is the name of the edge, assigned based on the order in which
    edges are listed in the input file. Notice that the structure of the
    produced graph does not depend on the edge ordering, however the id of the
    edges does.
    \item {\bf weight}: is the number of registers present on the edge.
    \item {\bf src}: is a pointer to the source vertex of the directed edge.
    \item {\bf dst}: is a pointer to the destination vertex of the directed
      edge.
\end{itemize}

Among the methods available in class \texttt{sng}, it is worth to mention:
\begin {itemize}
  \item {\bf reset\_deltas}: this method resets the arrival time of the
    vertices, which is necessary before running CP after the network has
    been retimed.
  \item {\bf retime\_sng}: this method takes as input a retiming vector
    and applies it to the graph. All edges weigths are properly updated.
  \item {\bf revert\_sng}: this method takes as input a retiming vector
    and reverts the network retiming.
    FEAS calls this step when a target clock cycle is non feasible. Also
    we revert to the initial graph every time we try a new target cycle.
    Being able to revert a retiming, allows us to avoid copying the graph
    while running the tool.
\end {itemize}

\subsection{WD calculation}

Matrices of the arrival times D and of the paths weights W contains a fixed
number of elements, which is the square of the number of vertices. Therefore
we choose to avoid using dynamic memory allocation ad to represent the two
matrices as two dual dimension arrays of size n by n.

The class dedicated to WD algorithm has also a vector of unsigned integers,
which is used to store and sort all possible values to be picked as target
clock cycle in \texttt{min cycle} mode. The values stored in this vector
are dumped from matrix D, then we sort the vector and remove all duplicates.

The methods of this class are the basic steps of the Floyd-Warshall algorithm:
\begin{itemize}
  \item {\bf initialize WD}: we initialize the matrix based on the \texttt{sng}
    graph. At the beginning, we only know vertices delay and edges weight.
    All paths not yet discovered are marked with an infinite weight and 0 delay;
    on the other hand, all paths corresponding to an edged are marked with the
    edge weight and the delay of the source vertex.
  \item {\bf compute WD}: we run a single pass of Floyd-Warshall \cite{algo} and we
    update while the algorithm runs both paths weight and delay, thus we
    find per each path the \texttt{min-weight - max-delay}.
    During the last iteration of the outer loop of Floyd-Warshall we add to
    the paths the delay of the destination vertex and we determine the
    current minimum clock cycle of the network.
\end{itemize}

\subsection{FEAS algorithm}

The class implemented for FEAS does not need any additional data structure.
It includes: a copy of the pointer to the network of type \texttt{sng};
a class, named \texttt{cp}, which includes the methods to run DFS on the
graph and determine the feasible clock cycle after each retiming step; a
set of auxiliary methods to run the kernel of FEAS.

To avoid copying the network, at each iteration of FEAS, we retime the network
starting from the graph obtained at the previous iteration. We also update the
elements of the retiming vector, with respect to the input network, but we
use a local retiming vector at each iteration, which represents the retiming
between iteration $i$ and iteration $i-1$.
Finally, if the retiming was successful, we return the global retiming vector
along with the retimed graph, otherwise, we revert the network before returning
to main.

\subsection{Linear programming: constraints generation}

While the tool is able to generate the tableau to run SIMPLEX in a single step,
starting from matrices WD and from the input graph, we added additional
methods to generate and sort the list of constraints to which our linear
program is subject to.

At first we add to the list of constraints the necessary conditions to get a
legal network, which implies that no edge has a negative weight.
Per each constraint added, we introduce a new slack variable to the system
and we add a row to the matrix.

As a second step, we add the timing constraints, derived from WD matrices.
This time, since from the matrices we get some redundant constraints, we
mark the required ones, so that we can print both a complete list of
constraints and an optimized one. However, we only add to the matrix the
reduced list of constraints, while we skip the others.

To support sorting and printing we use a vector of type \texttt{p2\_triple},
that includes the id of the vertices involved, the coefficients of the
disequation, the sign of the answer term and a flag to distinguish between
redundant and necessary constraints.

\subsection{Linear programming: SIMPLEX}

SIMPLEX algorithm relies on the representation of the tableau, which is a
vector of rows, where each row is a class that contains the following items:
\begin{itemize}
  \item {\bf label}: the variable for which the row tells the basic solution.
  \item {\bf coeff}: a vector of integers, which are the ordered coefficients
    of the disequation represented by the row. The first n coefficients are
    related to the retiming vectors elements (one per each vertex of the graph),
    while the subsequent numbers relate to the slack variables used for SIMPLEX.
  \item {\bf ans}: is the answer term of the row.
  \item {\bf ratio}: is a string updated at each iteration of FEAS, while
    searching for the right pivot in the matrix. This string is useful only
    in verbose mode to print intermediate results, while the actual ratio
    is not stored, because we determine the pivot with a single pass across all
    the rows.
  \item {\bf star}: is a flag that tells if the coefficient of the variable
    represented by the label of the row is negative. This flag is used in
    phase 1 of FEAS to manipulate the matrix in order to obtain a feasible
    solution for the linear program.
\end{itemize}

The class \texttt{simplex} contains a collection of methods to create the tableau,
by adding all the rows derived from the constraints. Moreover, the last row added
to the matrix corresponds to the objective function, determined based on the
information stored into the graph, and in particular, the \texttt{c} value of
each vertex.

Most important methods called in \texttt{simplex} are functions to find the
appropriate pivot column, to determine the pivot row and to clear the column
of the pivot with a simple row operation applied to the entire matrix.

It is worth to mention that in verbose mode only, the rows in the tableau are
ordered before starting FEAS kernel, so that each row of the initial tableau
corresponds to the disequation in the reduced set of constraints listed in the
same order.


\section{Testing Methodology}

Due to time shortage, we decided to design and test our program
incrementally, running small simple intermediate tests to check the correctness
of every algorithm or portions of them.

We produced some hand solvable inputs for Floyd Warshall, CP, FEAS, network
retiming and reverting, constraints generation step, objective function generation,
SIMPLEX and we ran all the tests separately while coding the related algorithm.
Notice that SIMPLEX was also tested against a web based applet that solves
linear programming using SIMPLEX method.

Thanks to the flexible data structures and the methods interface we provide, we
could reliably assembling the tool, starting from tested basic components,
with a low probability of introducing new bugs.

We also used compiler pragmas to enable two levels of degbug:
\begin{itemize}
 \item {INFO}: when defined we print all basic information of the algorithms,
   along with some intermediate results. This level of information is easy to read
   and allows to quickly identify at which step of the computation the error
   occurred.
 \item {DEBUG}: when defined we print detailed information of every function
   that makes some computation. The tool prints a huge amount of extra information,
   which most of the time allows to identify the cause of the error. It is
   recommended to get first the basic information using the definition INFO
   and finally to use the DEBUG information to dig into the problem
\end{itemize}

To test the entire tool in both modes we leveraged the two given examples of
the correlator. Also we tested the tool with a larger network, which was not
strongly connected. This test helped us to check if the tool behaves correctly
when WD matrices are incomplete.
All the three examples were compared against a manual solution of the retiming
problem.

Most of the issues we faced were cause by a slightly incorrect formulation and
set up of the problems, rather than by the implementation of the algorithms,
which are small and simple to code.


\section{Additional Insights}

In section \ref{sec:struct} we described in detail our choices in terms of data
structures that make our retiming tool fast and memory efficient.

It is worth to mention a few more features of the tool, which we implemented:
\begin{itemize}
  \item Reduced number of loops, obtained by computing all results
    on-the-fly, rather than iterating many times to produce incrementally
    the information. The drawback of this optimization is that we had to step back
    and massage the code, in order to print all the required outputs, which
    are sometimes irrelevant to the computation of the final result.
  \item Each vertex and each edge is ``self contained'': no need for iterating
    over the list of edges or over the list of vertices to retrieve the
    information needed, because each vertex has a pointer to each edge
    which impinges it. Similarly each edge has pointers to source and destination
    vertices.
    \item Binary search for \texttt{min cycle} prunes the search to values which
      are smaller than the initial clock cycle.
    \item Tableau rows labeling for \texttt{min area} uses numbers instead of
      strings and a member variable keeps track of each added row and slack variable,
      thus making easier to perform row operations and to read the basic solution
      from the tableau.
\end{itemize}

\section{Technical Issues}
\label{sec:tech}

\subsection{Floyd Warshall Algorithm}

\begin{enumerate}
  \item Iteration 0 $(k = 0)$: all paths not yet discovered are marked with an
    infinite weight (Matrix W) and 0 delay (Matrix D);
    on the other hand, all paths corresponding to an edged are marked with the
    edge weight and the delay of the source vertex. Also, edge weights between
    the same nodes is set to 0.
  \item Iteration 1 $(k = 1)$: At the end of first iteration,  matrix W has the
    following entries $wuv(1)$: $wuv(1) = min( wuv(0), wu1(0) + w1v(0) )$
    where $wuv(0)$ is the minimum weight between u and v vertices from iteration
    0, $wu1(0) + w1v(0)$ is the minimum weight between vertex u and vertex v,
    if vertex 1 is an intermediate vertex of the path between u and v. The second
    expression in the min function is again calculated from the minimum distances
    computed during iteration 0.
    Any entry updated in matrix W causes an update of the same entry in matrix D,
    which keeps track of the total delay between the two vertices uv, when the path
    with weight wuv(1) is selected.
  \item Iteration i $(k = i)$: At the end of ith iteration, matrix W stores
    the minimum weight between any two nodes u, v, if there exists a path from
    u to v of length at most $i$. Matrix D stores the total delay of such a path
    from u to v. Each element is calculated with the same expression shown
    for step 1, in which we have $i$ in places of index 1 and $i - 1$ in place
    of index 0.
\end{enumerate}


\subsection{Retiming: Handling too slow operators}

The timing constraints for the retiming problem  state that
\begin{equation}
  \forall V_{i}, V_{j} s.t. D(V_{i}, V_{j}) > \phi \ \ \ \ \  R_{i} - R_{j} \leq W(V_{i}, V_{j}) - 1
\end{equation}
If we have $D(V_{k}, V_{k}) (=30) > \phi (=20)$, then the following equation must be satisfied:
\begin{equation}
  R_{k} - R_{k} \leq W(V_{k}, V_{k}) - 1 \ \ \ \ \  W(V_{k}, V_{k}) = 0
\end{equation}
The above inequality will clearly never hold and therefore the system of equations  will always be unsolvable.


\subsection{Retiming: Exploring alternative retiming vectors}

\begin{enumerate}
  \item Legal retiming vectors $r_{x}$, larger than $r_{opt}$ are not always
    feasible.
    Referring to figure 9.8 in \cite{dm}, the following retiming vector gives a counter example:
    -[1 2 2 3 1 1 0 0]. Retiming value of vertex $V_{e}$ in figure 9.8 of \cite{dm}
    is increased from -2 to -1. Equation 9.4 in \cite{dm} is still satisfied, which implies
    a legal vector, but equation 9.5 in \cite{dm} is not satisfied for vertices $V_{e}$ and $V_{f}$.
    \begin{equation}
      R_{e} - R_{f} \leq W(V_{e}, V_{f}) - 1
    \end{equation}
    Computing the left hand side (LHS) of the above equation we have
    $-1 - (-1) = 0$ which is not less than or equal to $-1$. Hence the
    retiming vector is unfeasible.

  \item Similarly we provide a counter example for smaller retiming vectors.
    -[1 2 2 3 2 2 0 0]. Here retiming value of vertex $V_{f}$ is decreased
    from $-1$ to $-2$. Equation 9.4 in \cite{dm} is still satisfied, indicating a legal
    vector, but equation 9.5 in \cite{dm} is not satisfied:
    \begin{equation}
      R_{e} - R_{f} \leq W(V_{e}, V_{f}) - 1
    \end{equation}
    LHS: $-2 - (-2) = 0$, which is not less than or equal to $-1$.

  \item None of the smaller legal retiming vectors are feasible for the correlator
    example. In the five relevant inequalities corresponding to equation 9.5 \cite{dm},
    if we reduce the retiming values of the vertices that are used in these
    inequalities then we either get a retiming vector which is not legal or
    it does not satisfy the feasibility constraint.
    Looking at the variables $R_{i}$ and $R_{j}$ (corresponding to vertices in the
    five inequalities) in equations 9.4 and 9.5 in \cite{dm}, if we reduce the value of $R_{i}$
    then we either violate the legality constraint (for vertex 1) or violate the
    feasibility constraint. On the other hand if we reduce the value of $R_{j}$ then
    we will always violate the feasibility constraint.
\end{enumerate}

\subsection{SIMPLEX: selecting a pivot column}

\begin{enumerate}
  \item Choice of negative column: The goal of simplex algorithm in each iteration
    is to reformulate the linear program so that the basic solution has a greater
    cost (for a maximization problem). Following this principle, the pivot column
    selected contains a negative number in the bottom row, as the bottom row
    represents the objective function and the negative number is actually a positive
    coefficient of a non-basic variable in the original objective function.
    Increasing the value of this non-basic variable will lead to an improved cost
    during each iteration.
  \item Other negative columns:
    {\bf Iteration $i$ vs $i-1$}: The cost of the current iteration will be higher as compared
    to $i-1$. This is because a non-basic variable is selected which has a
    positive coefficient in the original objective function.
    {\bf Iteration $i$ (alternative negative pivots)}: The objective cost for the
    current iteration will be higher if we use greatest magnitude negative pivot
    column. This is because the greatest magnitude negative column corresponds to
    the greatest positive coefficient of original objective function.
\end{enumerate}

\bibliographystyle{plain}
\bibliography{ref}

\end{document}
